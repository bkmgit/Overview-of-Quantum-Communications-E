\chapter{BB84: Single-photon QKD}


\section{Three phases of cryptographically secure communication}

00:00
hi and welcome to lesson nine this lesson will be on a single photon qkd
and particularly on the first single photon qkd protocol known as bb84
so step one what are the three phases of cryptographic secure communication
so secure communication proceeds roughly in the following
three phases first the parties need to authenticate each other meaning that
they and they they identify to each other that they are really the parties
that they want to communicate with and not somebody else
then they have to generate a key that they will use for encoding their message
after the key is generated they can get into encoding their message and encrypt
their data send it to the other party where it will be decrypted

00:01
so authentication between alice and bob usually proceeds in the following
following way so first alice says that i'm alice to bob then bob can say okay
you're alice i'm bob then also anybody could say that so the question
is why should i believe you why should i trust you why are you really the person
that you're claiming to be and bob of course has the same question
so generally the message and the procedure how
two parties can authenticate each other is
who are you what do you have and what do you know
and in particular the last one is quite important
because you know that the party that you are trying to communicate
knows something for example like a public key or a pre-shared search secret
and that can be used as an authentication device

00:02
so after authentication they have to generate a key now the key is then
will be subsequently used for encrypting the
message and the key can be for example there are many different
ways of exchanging keys such as diffie-hellman key exchange
is one such method but the problem with the key is that once it is generated it
can be used but it cannot be used forever it always has to be changed
and it has to be changed frequently in order to ensure
that the data is encrypted in a secure manner so once the key is generated alice
can encrypt her message with the key she can send it to bob
where bob will use his generated his part of the generated key
to decrypt the message and read it and common forms are are the following ds

00:03
3ds aes less common one is one-time pad and we will describe what one-time pad
is sometimes it's known as the vernam cipher

\section{Private vs public key}


00:00
step 2 private versus public keys so one way to establish a private key
between alice and bob is to use public channel
and keys that are established in such a way are known as public keys
so here's how it works we've got two parties that are trying to communicate
alice and bob and they can communicate over a public channel
public means that anybody has access to this channel so any message that they
transmit that they used to communicate with can be heard and intercepted
by any other party so this channel is not secure in the sense that anybody
can read any message that they sent so what alice does is she generates two
keys one is known as the public key and the other one is known as the
private key what she then does is she sends the public key to bob that she

00:01
generated he uses that to encrypt his message and then what he does is he
can just send it back to alice then she uses the private key
that she did not communicate to bob to decrypt bob's message
and read what it is so in public key cryptography
public key is used to encrypt the message but it cannot be used to decrypt the
message the message is decrypted by the private key that alice
generated but did not share with over the public channel
and it works but it has some disadvantages it is slow it is expensive and more
importantly it is what's known as computationally secure
meaning that anybody listening to the channel is able in principle to break the

00:02
encryption if either they have a large amount of computational resources
or computational time so it is not unconditionally secure
a different way of establishing and this is the crucial point is that
quantum computers can in principle break computationally
secure protocols like this with relative ease so how can we how can we actually
establish secure connection between alice and bob
the other alternative method is to use a private key
here what happens is that there is some generator some hypothetical device that
can generate private key and that key is then sent to alice and sent to bob
now alice and bob are sharing some correlated secret key that only they know
and they can use it to encrypt their message

00:03
for example bob encrypts his message he sends it to alice
where alice uses her part of the secret key to decrypt it and read the message
this is known as one time pad or the vernam cipher
and it is secure it cannot be broken provided that is used only once so
if bob has a message of n bits that he's trying to send to alice
he requires a private key that's at least n bits long
and once he uses that private key to encrypt his message
he cannot use it again if he has some other thing to say to alice
they require a completely new and fresh private key to ensure security
so as we said it's not very efficient in the sense that it
requires large amount of key bits the number of keys has to be at least as
high as big as the message itself so that's actually pretty good but

00:04
there is one remaining question and that's uh
how do we actually distribute this key in this scenario we just assume
that the private key generator can distribute this private key to alice and bob
but of course there is a question of what channel should the private key
generator use because public channels are monitored by
whoever is listening and this is where quantum mechanics comes into
play as we will see in the next step you

\section{BB84 Protocol}


00:00
step 3 bb-84 protocol this protocol is a quantum protocol
therefore alice and bob can utilize a public quantum channel as well
as their public classical channel so here we've got alice and bob again
and they can communicate over this public quantum channel what alice does
to begin the protocol is she generates two n-bit strings
we're going to call the first n-bit string a
so there's n bits in her string a we're going to denote them a1
a2 up to a n and the second bit string we're going to denote is b and there's
b1 b2 all the way up to bn and then what she does is she creates quantum states
according to these bit strings as follows for each two bits from a and b

00:01
she takes them and creates one qubit and then she creates n such qubits and
the whole state will be denoted by psi so if her qubits a k and b k are 0 0 then
she encodes this qubit she prepares the qubit in the state 0.
if ak and bk are 1 0 then she prepares state 1. similarly if they are 0 1
she prepares state plus and if they are one one she prepares the state minus
so we can see that the bit coming from the bit string b determines the basis of
alice's if it's zero as we can see in both of these cases here and here she
prepares the qubit in the z basis if it's one as we see in this case and in this
case then she prepares the qubits in the x spaces

00:02
and ak then chooses which state from the basis she prepares if a is zero
she prepares the plus one ideal state if it's one she prepares the minus one i
guess state notice that these states are not orthogonal
for example if we take the inner product between psi 0 0 and psi 1 0 psi 0 1
like here we see that they are not orthogonal meaning that their inner
product is non-zero in this particular case it's 1 over square root of 2.
same if we take for example the state psi 1 0 and psi 1 1.
again we will get that the inner product is non-zero
when this happens when the inner product
is non-zero it means that the two states are not perfectly distinguishable
and this is crucial ingredient in this protocol
so what does it mean for two states to be non distinguishable

00:03
let's consider the case where we are measuring in the z basis
and we are given two states one state is the zero state
and the other state is the one state if this happens
we see that they are orthogonal it means we can perfectly distinguish them
so if we just keep measuring the incoming qubits
if the qubit is in zero we will always get a plus one outcome
on the other hand if the qubit is in one we will
always get minus one outcome so with certainty we can distinguish whether the
incoming qubit is a zero or a 1. same thing applies
if we are measuring in the x basis and we are given only state plus
or state minus like here if we keep measuring
x and we get this state then it will the outcome will be always
plus one with hundred percent um probability if on the other hand we get a minus
state and we measured in the x we will get outcome minus one all the

00:04
time so in this sense we can distinguish plus and minus
if we're measuring the x bases on the other hand let's say that we are
given this following states zero or a plus and we measure in the z basis
so if we measure the zero state like we said we always get the plus one outcome
that's fine but if we measure the second state the plus state
there's a fifty percent probability that we get the outcome plus one
and fifty percent probability that we get the probability
that we get the outcome minus one so if we get a state zero all is good but
if we get the state plus sometimes we will get a plus one and
sometimes we will get a minus one if we get the minus one state fine we
can say we know that this state is plus but if we get the plus one outcome we
are unsure whether the state is a plus or whether it's a zero
in this sense the states are non-distinguishable

00:05
we can do the same thing in the x spaces and here the scenario is reversed
we can all we with certainty we can say that we get plus one
outcome here but if we get in the x if we measure in the x basis and we
receive the state 0 sometimes it will be a plus 1 and sometimes a minus 1.
so let's consider an example of the encoding
let's say that alice generates string a which is 0 1 1 0 1 and string b
1 1 0 0 1. so then she starts encoding she looks at the first
bit in her string b which is one so she knows now i have to encode in the x
spaces and the state that she prepares is a plus state
because uh her first bit in the a string is a zero
so that's her first qubit that she prepares the second qubit
she looks at the second bit in her string b which is again one

00:06
so again she knows she has to prepare a state from the x basis
and the state is given by this uh second bit in string a which is one
therefore she prepares the state minus and then she goes on and on until she
prepares all n qubits what she does then is she sends these qubits to bob
over the public quantum channel now let's consider what bob knows at this time
he actually doesn't know what these states are
because alice did not share the secret uh string b containing information about
the preparation basis with anybody she kept it secret
therefore all bob knows is that he's receiving qubits and they can be
any of the four possible states 0 1 plus minus
but what he does he goes on anyway and he creates his own
random bit string which we are going to denote as

00:07
b prime and again because uh he's expecting n qubits
he generates n uh bits b one prime b two prime on
all the way until b n prime and what he then does
he just measures in the basis given by this bit string so
similarly to alice if b prime if the bit b prime k is zero then he
measures in the z basis and if it's in one then he measures in the x basis
this allows him to generate his own random bit string which we will denote a
prime so if the outcome of the k-th measurement
on the qubits is plus one then he will assign zero to a bit a prime k
if it's minus one then he assigns a bit one to uh uh a prime k

00:08
and this way he generates his own a prime random bit string
what then happens is that alice and bob they share information with each other
over a public classical channel alice shares her randomly generated bit string n
and bob shares his randomly generated bit string b prime
so they exchange information about the basis in which the qubits were prepared
and in which they were measured and what they do is if they measured and
prepared in the same basis they will keep the corresponding bits from
a and a prime if they measured in different bases then they just discard
the bits a and a k and a prime why because as we said if alice prepares in
a certain basis and bob measures in that basis the two
states the two possible states are orthogonal meaning they are

00:09
perfectly distinguishable by measurement it does basis so it allows
bob and alice to generate a perfectly correlated
key so the so the bits that they keep we're going to denote as a uh bar and a
bar prime and they are perfectly correlated meaning that these two shorter bit
strings that they generated are in fact equal
and now they are sharing a key that they can use
in the next step which is encryption of the data
so for example how does it work let's again consider case where
n is equal to five so alice has randomly generated five bit strings a
and b b tells her in which basis she should encode a tells her what the
state should be so she prepares the following states minus zero plus
minus one bob generates his own random b string b prime again this

00:10
bit string is different from alice's b because he doesn't know alice's b at
this time and he just measured in the basis given by b
prime so here the first bit in b prime is one therefore he measures in x
on the second bit it's again one therefore he measures in x
and so on and so forth and you can see that since
in the first for the first qubit alice encoded the qubit in basis x
and bob measured in basis x therefore bob will
always with hundred percent probability get the outcome
minus one which we said corresponds to uh bit string a
prime which is equal to one and we see that in this case
a prime one is the same as alice's a one on the other hand
in this for the second qubit alice prepared the qubit in the z basis

00:11
but because bob's uh random bit b prime two is given by one
he measured in the x basis so with fifty percent probability he can obtain zero
with fifty percent probability he can obtain one
so they cannot be sure that they are really sharing a correlated
bit therefore they discard this bit for the third qubit again the the alice
prepared in the same basis as bob measured therefore they know in this place
they can share they can keep this bit and use it as part of their secret key
and this way they can generate shorter keys
but which are 100 correlated so here we see in the first for the first bit they
keep it second one they discard third one they keep it
fourth one they discard and the last one they also keep
so the shared key that they have is given by one zero one
so here's the summary of the protocol so far

00:12
alice starts by generating two n-bit strings a and b b is used for
basis of the measure of the preparation whereas
a tells which state of that particular basis alice should prepare the qubit in
if b k is equal to zero then she prepares in the z basis
if b k is equal to one she prepares in the x basis
then alice sends these qubits to bob over a public quantum channel
bob measures randomly either in z or x spaces
after the measurements are completed and only after that
they are allowed to share the information about the preparation basis
and the measurement basis and they keep the bits only where they prepared and
measured in the same basis again they do that because they are guaranteed
that in this scenario the results that are generated are 100

00:13
correlated and whatever is left of the key they use as a secret key for
encoding and decoding their messages now so far we have only considered the
ideal scenario where there was no eavesdropper
now let's say that somebody is listening to both the public classical channel
but also the public quantum channel so we are going to consider the effect of an
eavesdropper which we are who we are going to name eve
and see uh what effect that has on the protocol and how the protocol can
actually discover the presence of such an eavesdropper


\section{Eavesdropper detection}


00:00
step four eavesdropper detection in the previous step we have seen how
the ideal protocol works now let's see what happens when we
include the effect of an eavesdropper trying to gain
access to the secret key that alice and bob are trying to generate
so we've got the following scenario we've got our alice communicating over a
public quantum channel with bob and there is some third party eve
that wants to intercept their messages so what can what can eve do
let's assume that she can take these qubits that alice is sending over the
public quantum channel to bob she intercepts them can she copy them
and then resend them to bob that would give her access to the protocol
and the qubits and also subsequently the secret quantum key
luckily she cannot do that due to the no cloning theorem

00:01
as we have seen in the previous lesson in the teleportation lesson so
she cannot simply take them make a copy of the qubit
and then resend the qubits to bob that's good so in order for if to gain
any access um and to the information that alice is trying to share with bob
she has to measure these qubits so what can happen then she has to guess
the basis of measurement because remember the
preparation basis stored in the b string b
is still kept a secret by alice that has not been communicated over a public
classical channel to bob therefore alif does not have access to this information
so in order to measure these qubits she just has to pick a random basis
but she knows that the basis the the two choices for the basis is either z or x

00:02
so if she measures in some basis this has a chance to disturb the qubit
for example if alice prepares the qubit in the x basis
and if measures in the x basis then we have seen that in such a scenario
there is no disturbance the qubit is still projected onto the same state that
it was prepared in so it doesn't change the state of the
qubit and it doesn't change the basis of the qubit
however in the second case let's say that alice's preparation basis is zed
and eve's measurement basis in is x so the qubit originally prepared by
alice was either in a zero state or a one state these are just two
possible states in the z basis but by measuring them in the x basis
alice is now projecting onto the either plus state
or a minus state so she disturbed both bases and the state similarly in

00:03
this scenario here where alice is prepared alice prepares in the x basis and eve
measures in the z-basis and this is the basis this is the main
principle how alice will be detected by alice and bob
so if she guesses wrong if if guess is wrong
she will change the basis of the qubit so let's see what happens alice prepares
her qubits she starts sending them over the public quantum channel
eve intercepts these keywords and she measures them in a random basis
and then she passes on these qubits to bob so some of these cubits where she
guesses correctly so she measures in the preparation basis
they do do not get disturbed but some of these qubits represented by
these black qubits they become disturbed this is when alice

00:04
measures in a different basis than the preparation basis
then this gives alice and bob an opportunity
to detect that there is an eavesdropper present how does that happen
basically what they need to do is they go through with their
bb-84 protocol they compare their bases in which alice prepared and both
measured they keep only those bits where they measure the qubits in
the same basis and then they dedicate a portion
of this new shorter key to finding to detecting an eavesdropper eve
so eve has a chance of measuring in the same basis as the preparation basis
of one half so fifty percent and also bob chooses the same basis as alice with
probability fifty percent so this gives alice and bob a
probability of one quarter to detect an existence of if because remember

00:05
if alice prepared in one basis and both measured in the same basis
they are expecting this classical measurement outcomes to be the same
in they are expecting to share a perfectly correlated key
but if alice measured this qubit and disturbed this qubit then there is a
chance that these two bits will not coincide and if they don't coincide then
bob and alice know that something went wrong and there's a pr
ease dropper present trying to gain access to their secret key
so we said that alice can be detected with probability one quarter
when it comes to measuring a single qubit when there's
n qubits that she measures the overall probability
of detecting alice is given by the following expression
one minus three quarter to the power of n so let's see how this scales with the
number of n here it is on the horizontal axis we're

00:06
plotting the length of the bit string that we dedicated to detecting
an eavesdropper if and on the vertical axis we've got the probability of
detecting if so you see as the number is very small so we don't
small n is very small we don't dedicate too many bits to actually trying to
detect the eavesdropper then also the probability that ev's
detective is relatively small but very quickly this probability shoots
very close to one so with nearly a probability of hundred percent alice
and eve can detect the presence of an eavesdropper
even for n equals to 25 so if they dedicate 25 bits to detect alice
the chance is one in a thousand that if will not be detected
so it's very very small meaning with very high probability alice and bob
know that somebody's eavesdropping onto their channel so
they can just say we know that we are not sharing a

00:07
secure secret key therefore we choose not to continue
with the rest of our communication and we don't allow eve to gain access to our
sensitive message

\section{Existing QKD network testbeds}


00:00
step five existing qkd network testbeds in the previous steps we have seen how
alice and bob can use quantum mechanics to discover an eavesdropper and generate
a secret random key for communic for secret communication
in this step we will demonstrate that actually these things are not only
in the realm of theoretical research but also have been tested
uh by building real networks so the very first network that was built
was known as the darpa qkd network all the way back in 2004 and this included
10 nodes in a network and it was built in the state of massachusetts in the usa
and in particular it included different physical links for example
here you had a communication over free space here you had a communication
over over a fiber so this was unlike any of the previous experimental

00:01
implementations where secure quantum communication
in the form of a single photon based qcd was done
only on a point-to-point basis over a single link
this was the first network experiment demonstrating the viability of quantum
key distribution the next experimental effort was in
in europe known as the seacock qkd network
seacock stands for secure communication based on quantum cryptography
and it was around the year 2008. it launched
in vienna and it comprised of six nodes and eight links and involved the layered
architecture and finally there was a there are many
other experiments but one notable one was done in tokyo around
2010 and it was known as the tokyo qkd network here you can see the

00:02
physical map actually which places in tokyo were connected
you have hunger campus here in tokyo university
you've got otemachi over there and it reached all the way to nict in kogane
and the interesting part about this network that actually it was used
for a secure quantum secure video conference which really demonstrate
the viability of using quantum mechanics for securing communication in a network



\newpage
\begin{exercises}
\exer{Consider the following quantum state:}
\begin{equation*}
\ket{\psi} = \frac{\sqrt{3}}{2}\ket{0} + \frac{1}{2}\ket{1}
\end{equation*}
\subexer{Find the probability of measuring a zero.}
\subexer{Find the probability of measuring a one.}


\end{exercises}

