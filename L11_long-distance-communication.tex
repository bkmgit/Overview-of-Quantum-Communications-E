\chapter{Long-distance communication}
\label{sec:11_long-distance}

In this chapter, we will give you the context of how we can communicate over long distances in classical networks, and see how that fits within the context of quantum communication as well.

\section{Introduction}
\label{sec:ld-intro}

% Hi, and welcome to Lesson Eleven on Long Distance Communication.

Let's go back to the year 1852 and consider a little bit of the historical background of long distance communication.

\begin{table}
\begin{tabular}{|c|c|}
\hline \multicolumn{2}{|c|}{ Letter from London ${ }^*$} \\
\hline to & took [days] \\
\hline New York & 12 \\
Bombay & 33 \\
Singapore & 45 \\
Sydney & 73 \\
\hline
\end{tabular}
\caption{Prior to the invention and deployment of the telegraph, long-distance communication was slow.}
\label{tab:london-letter}
\end{table}

Just before the widespread deployment of the telegraph, communication was extremely slow. To give you some idea of how slow it was, if you wanted to send a letter from London, let's say to New York, it took around twelve days to arrive, as shown in Tab.~\ref{tab:london-letter}. If you wanted to send it to Sydney, then you had to wait seventy three days for the letter to be delivered. The first long-distance telegraph line, made of copper wire, stretched over land from Washington to Baltimore, but the telegraph immediately sparked the dream of a transatlantic submarine cable. People realized that if they can use the telegraph to communicate over land so quickly, they should be able to connect continents with submarine cables as well. So in 1850 the very first submarine telegraphic cable was laid, connecting England and France. That worked fine, so immediately people wanted to lay a cable across the Atlantic Ocean. For the next fifteen years there were many failed attempts, but this sparked a focused mathematical analysis of very long distance communication. \rdv{tell me more...?}. Although one cable functioned long enough for Queen Victoria of the United Kingdom and President Buchanan of the United States to exchange messages in 1858, it wasn't until 1866 that the first truly successful transatlantic telegraphic cable was laid and used~\footnote{This preceded the first transatlantic radio communication by thirty-five years.}. By 1871, only few years later, all the continents except Antarctica were connected.

Then, between 1902 and 1906, the first transpacific cables were laid, connecting mainland US with Hawaii, Guam, and later Philippines, and finally Japan~\footnote{rdv finds it an interesting historical anecdote that one of Japan's first undersea cables came ashore in the seaside town of Kamakura, where he lives, in 1931.  It connected to Midway, Hawaii, then the continental U.S.  It was in use for around a dozen years.}. After the telegraphic cables came telephone cables, and in 1956 we had the first transatlantic telecommunications cable called TAT-1, and this telephone cable was able to handle 51 consecutive telephone conversations. Compare that with the state two decades later, when the last of the \rdv{dedicated? analog? coaxial? non-fiber?} telephone cables, TAT-7, which could handle a staggering eight thousand telephone conversations concurrently. And after this came the fiber optic cables. As we saw in the previous lesson, in 1988, one of the first fiber optic cables called TAT-8 was laid and increased the bandwidth to the equivalent of forty thousand telephone conversations.

\rdv{maybe one paragraph here on baseband v. modulated signals, channels and filtering?  Otherwise it's not clear why many conversations or connections can be handled in the same cable.}

In this chapter, we are going to focus on the main considerations when designing these cables, particularly bandwidth and noise.

Bandwidth tells us basically how much information a cable or fiber can carry.
This depends on both the physical properties of the fiber itself, as well as how clever we are when it comes to encoding this information before it gets transmitted. With modern techniques such as dense wavelength division multiplexing, we are able to reach some staggering speeds. For example, in a standard cable over the distance of 6,600 kilometers, we can reach speeds of something like 65 terabits per second. In a more specialized cable and over shorter distances, this can be increased to over 150 terabits per second. Over very short distances, we achieved speeds of one petabit per second. This is something incredible! (A terabit, just for your reference, is $10^{12}$ bits, while a petabit is $10^{15}$ bits.)

Of course, no system is a hundred percent efficient, and what we put in is not exactly what we're going to get out at the end of the cable. We must consider the main sources of loss in optical fibers. We will consider the following five losses: dispersion, absorption, scattering, bending, and coupling. The combined result of all of these sources of loss is that our signal becomes attenuated, meaning if we put some signal as an input with some power, what we get out as the output will have less power. It's very important to know how much of the signal becomes attenuated and how we can prevent this attenuation or combat this attenuation.

In this chapter, we will first talk about the dispersion in optical fibers, then we will consider the other sources of attenuation. Then we will move on to overcoming these losses, and finally we will consider the challgnes these sources of losses present in the context of quantum communication.


\section{Mode dispersion}
\label{sec:11-2_mode_dispersion}

Mode dispersion\index{mode dispersion} is the first source of signal degradation in the fiber that we're going to consider.

Let's consider the propagation of different modes in a multimode fiber\index{multimode fiber}. As we said, a multimode fiber can contain many modes, all traveling with different paths. For example, you can have the axial path which travels directly down the fiber, or you can have other modes which are  totally reflect internally within the fiber like that (see image). And as you can see the different modes, they propagate at different speeds because they have to traverse different lengths. So it's very important to compute what is the time difference introduced by the different modes traveling at different speeds down the fiber, and this as you can expect, depends on the launch angle, so depending on the angle with which it is coupled to the fiber, the path that it takes will be different, and therefore the time that it takes to traverse the fiber will be also different.

The fastest mode, as you can clearly see, travels directly down the center of the fiber. This is known as the \emph{axial mode} because it travels down the axis of the fiber. The slowest mode is the one that is incident on the cladding just at the critical angle, meaning it just gets internally reflected.  Let's compute the time delay between the fastest mode and the slowest mode.

Your initial digitized signal may look something like this (fig. 2). It's very sharp and it's very easy to read out when you have a zero and when you have a one. But as the different modes propagate down the fiber, the whole package will spread and disperse. It will look something like this (fig. 3).

This means that the readability of the output signal worsens, meaning it's more easy to make a mistake when you're actually trying to decode your signal after it propagates through the fiber.

Let's start calculating the time delay between the fastest mode and the slowest mode. We're going to consider some length $L$ that the fastest axial mode traverses, and the time that it takes for this mode to traverse this distance, we will call $t_{\mathrm{min}}$.

We can very easily compute this value.
\begin{equation}
\begin{aligned}
t_{\min } &=\frac{L}{v_f}, \quad v_f=\frac{c}{n_f} \\
t_{\min } &=\frac{\operatorname{Ln}_f}{c}
\end{aligned}
\end{equation}
This $t_{\mathrm{min}}$ is just the distance traversed over the speed of light in the fiber, and we have seen that the speed of the light in the fiber is determined from the refractive index of the fiber material.
%So v-f is equal to c over n-f. We can substitute that into our expression for $t_{\mathrm{min}}$ and we obtain the following: $t_{\mathrm{min}}$ is equal to length $L$, times the refractive index of the fiber, all divided by the speed of light in vacuum c.

Now let's consider the time that it takes for some non-axial mode to traverse some distance $L$. Here (see slide), the distance that it travels is not actually $L$, but it's some other $l$. This will be the portion before the internal reflection $l_1$ plus the distance traveled after the internal reflection $l_2$, giving us $l = l_1 + l_2$.

Let's start by computing $l_1$,
\begin{equation}
l_1=\frac{L_1}{\cos \theta_r}.
\end{equation}
using $\theta_r$ as the angle of reflection. It should be easy to see that $l_2$ is similar, giving us
\begin{equation}
l=\frac{L_1}{\cos \theta_r}+\frac{L_2}{\cos \theta_r}=\frac{L}{\cos \theta_r}
\end{equation}

So the angle over here (see pointer), we're going to denote as "theta r" for angle of refraction. And we know from basic trigonometry that this length (follow pointer) small l1 is equal to this $L$1, divided by the cosine of the refraction angle $\theta_r$.

That gives us l1, now how do we compute l2? Well, we can use a nice neat little trick, and we can actually reflect this arrow here (see pointer) which describes our distance l2 over there, so we are extending the distance like this (dotted arrow),

and then l2 is given just as $L_2$ divided by cosine of the refraction angle $\theta_r$. So, we can just sum them together and we see that the path length that the light ray traverses (small) "l", is just the sum of these two terms small l1 plus small l2. So it's given just as $L$ over cosine of $\theta_r$.

So we see that the path length of the light ray only depends on the initial angle described by $\theta_r$ and the axial length, $L$, over here (see pointer).

So from that, we can also compute the time that it takes to traverse this length small l, and again it's given by small l divided by the speed of light in the fiber, which substituting all for l and for v-f, we obtain this following expression: so it's the axial length $L$ times the refractive index over fiber, all divided by the speed of light in vacuum, times the cosine of the refraction angle $\theta_r$.

\begin{equation}
\begin{aligned}
t &=\frac{l}{v_f}=\frac{L}{v_f \cos \theta_r} \\
&=\frac{L n_f}{c \cos \theta_r}
\end{aligned}
\end{equation}

So, let's get back to computing $t_{\mathrm{max}}$.

$t_{\mathrm{max}}$ is the time to travel a critical non-axial path. As we said, that such a path where we are just being reflected internally in, meaning that the angle of incidence on the cladding is theta c, our critical angle, and we have seen in previous lessons that sine of theta c is related to the cosine of the refraction angle over here (see figure). $\cos \theta_r=\sin \theta_c=n_c / n_f$

So $t_{\mathrm{max}}$ can then be computed as (small) l times the refractive index of the fiber squared, over the speed of light in vacuum, times the refractive index of the cladding. 
\begin{equation}
t_{\max }=\frac{L n_f^2}{c n_c}
\end{equation}
Now, it's very easy to just put the expressions for $t_{\mathrm{max}}$ and $t_{\mathrm{min}}$ together, and we obtain the time delay between the fastest and slowest pathos or modes, 
\begin{equation}
\begin{aligned}
\Delta t &=t_{\max }-t_{\min } \\
\Delta t &=\frac{L n_f}{c}\left(\frac{n_f}{n_c}-1\right)
\end{aligned}
\end{equation}
Now, we are going to plug in some numbers and see why this time delay is actually important.

Let's consider a particular example where the refractive index of the fiber is given by $n_f = 1.500$, and the refractive index of the cladding is a little bit less, $n_c = 1.489$. We can plug these numbers into our previous expression for the time delay, and obtain that the time delay per kilometer as follows,
\begin{equation}
\frac{\Delta t}{L}=37 \mathrm{~ns} / \mathrm{km}.
\end{equation}
This doesn't seem like a very long time delay, but let's see how big the effect can be. The speed of light in this fiber is given by $c$ divided by the refractive index of this fiber, which is just $v_f=c / n_f=2 \times 10^8 \mathrm{~m} / \mathrm{s}$.

This time delay between the fastest and slowest modes means that our pulse spreads over a distance as it travels through the fiber. We can quantify this dispersion and obtain the value $7.4$ nanometers per kilometer. Every kilometer that our signal travels, it becomes more and more spread by this distance, $7.4$ nanometers. As we said, this reduces the readability of the output signal. Because our packages are becoming more spread out, we are losing the sharpness of our signal and it becomes more difficult to decode. In order to be able to read our output signal, we may demand that the pulses that are coming out of our fiber are separated by twice the value of the spread.

This means that in order for the dispersed signal that's coming out of the fiber- the pulses to be separated by 14.8nm, we require that the input pulses are also separated by at least 14.8 nanometers. This inadvertently places a limit on how fast we can transmit information because the pulses have to be separated by a certain amount, so we cannot place them more closely together and we cannot send the information at a higher frequency. So dispersion has a direct consequence on the frequency of the input signal, meaning that it also limits our bandwidth.



\section{Attenuation}
\label{sec:11-3_attenuation}

In this section, we will consider the rest of the sources of losses that we listed in Sec.~\ref{sec:ld-intro}.

First, let's look at \emph{absorption}\index{absorption}.

Absorption happens due to interaction of light with the material of the fiber. There are two types of absorption:  \emph{intrinsic absorption} where even if we have a perfect fiber, the intrinsic absorption is responsible for attenuating the signal. This is something that we cannot get rid of. It's due to the interaction between the photons of the signal and the electrons in the material. The electrons absorb the photons from the signal and they become excited \rdv{what happens after that? How does this differ from the scatter below?  It's a little unclear.}, therefore the overall  power of the signal decreases. This is just an intrinsic property of the fiber and we cannot really affect it, it's something that we just have to live with and accept. But luckily, it's not a very significant source of error, especially when we compare it to the other type of absorption, which is \emph{extrinsic absorption}.

Extrinsic absorption is due to impurities that are present in the fiber, and these impurities are introduced during the manufacturing process. On one hand, this is a much more significant source of absorption, and therefore attenuation of the signal, but on the other hand we can control it by perfecting our manufacturing process. Generally during manufacture, the manufacturers are trying to keep the the amount of impurities that are present in the fiber below one percent. \rdv{flaws such as bubbles in the glass or non-smooth interface between the materials?}

Another source of attenuation is \emph{scattering}. This is similar to absorption, but the light is not only absorbed, it is also re-radiated and re-emitted back into the fiber in random directions. This is done again due to the impurities in the fiber, and can be controlled by the manufacturing process and limiting the amount of impurities that are present in the fiber. There are many different types of scattering, such as linear scattering and non-linear scattering, but the details will not be discussed in the current module.

The final two sources of attenuation are \emph{bending} and \emph{coupling}. Bending is when we actually physically bend the fiber. Remember, we said that the angle of incidence is crucial for total internal reflection to take place, and therefore for the signal to propagate down the fiber. If we bend it, we can have a situation where at first, the light ray is traveling down the fiber totally internally reflected but then it hits the angle at the bend and suddenly at the interface it is reflected in such a way that it cannot satisfy the condition for total internal reflection anymore, and it just gets absorbed or it leaves the fiber.  Generally, the manufacturers specify some minimum bending radius, typically around ten to twenty times the diameter of the fiber.

The other source of error is the coupling error. Inevitably, we will have to join two fibers together, and if there's a gap between the fibers, then the light can of course escape through the gap, or another source of coupling error is when the fibers are not aligned together, so even if there is no gap but they are slightly misaligned like here in this bottom example (see slide), then light is allowed to escape and leave the fiber, therefore the overall signal will become attenuated.

So, those were the main sources of losses, now we're going to discuss how to quantify the attenuation in a fiber.

As the signal propagates through the fiber, it loses power, so we need to quantify how much power it lost. This is done by a unit called the \emph{decibel}\index{decibel}. The decibel designates the ratio of the two power levels: the power in and the power out.

The number of decibels is defined as the following expression:
\begin{equation}
\# \text { of } \mathrm{dB}=-10 \log _{10} \frac{P_o}{P_i}
\end{equation}
where $P_i$ in the power in and $P_o$ is the power out. Remember, the power out has to be less than power in, because the signal is getting attenuated.  Because this ratio is less than one, the logarithm is negative, so applying a minus sign lets us talk in terms of positive numbers. These tens are here because we are talking about decibels, "deci" means ten, so therefore we are multiplying by ten over here, and sets the overall scale. Now why do we have this logarithm here? And that's because we will be considering a large span of orders of magnitude between the power in and the power out, and therefore if we take the logarithm, then it will produce a sort of generally nice scale for the number of decibels.
\begin{table}
\begin{tabular}{r|c}
$P_o: P_i$ & $\mathrm{~dB}$ \\
\hline $1: 10$ & 10 \\
$1: 100$ & 20 \\
$1: 1000$ & 30
\end{tabular}
\caption{Examples of calculating loss in decibels (dB).}
\label{tab:decibels}
\end{table}

For example, if we have the ratio of power out to power in as $1:10$, meaning that ninety percent of the power is lost to attenuation, then this corresponds to ten decibels (10dB), as you can convince yourself by substituting for $p_0/p_i$ into this formula (on slide). If the ratio is $1:100$, then this corresponds to twenty decibels (20dB), if it's $1:1000$, this corresponds to thirty decibels (30dB). So you can see that this ratio of the power out over power in, is getting smaller by an order of magnitude, whereas here (see pointer) the increase in terms of the decibel is just linear and this is due to the definition in terms of the logarithm for the decibels.

We can also define the attenuation parameter $\alpha$, and this is the number of decibels per kilometer. So $\alpha$ is just our previous expression divided by the length of the fiber $L$. And we can now rearrange it, bring $L$ to the other side, bring the minus to the other side and divide by ten to obtain this expression (eq. 2), and we can also get an expression for the fraction of the power out over power in (red box) being equal to ten raised to the power of negative $\alpha \times L / 10$.

We said in the previous lessons that in 1970, the optical fiber managed to transmit $1\%$ of the power that was put in over a distance of one kilometer. We can now just plug it into this formula (see pointer)- well this formula over here (eq. 1), and we will obtain that $\alpha$, the attenuation parameter, is twenty decibels per kilometer (20 dB/km). Two decades later, the transmission rose to around $96\%$ over a kilometer, which corresponds to an attenuation level of 0.18 dB/km.

So, let's plot these two values to see what they look like. Here (see figure), on the horizontal axis, we are plotting the length of the fiber through which our signal is traveling, and on the vertical axis we've got the ratio of the power out over power in. So, if the length of the fiber is zero, then of course the ratio is one. Our signal hasn't traveled anything, so it didn't have chance to be attenuated. But as it travels through, this blue line corresponds to the attenuation levels that were achieved in 1970, so the $\alpha$ is set to be twenty decibels per kilometer, and this orange line is the attenuation parameter in 1990. So we can see that how quickly the ratio approaches zero for the very large attenuation parameter, $\alpha = 20$, whereas for the very low attenuation parameter it decreases much more slowly.

Now the question is knowing what the main sources of loss are, and knowing how to quantify them, how can we protect against these losses, how can we counteract these losses in our fiber.



\section{Overcoming losses}
\label{sec:11-4_overcoming_losses}

We have seen what the sources of losses are. Now, let's discuss how we can overcome them.

Let's look at dispersion first, that's probably one of the easiest. We saw that the signal and the different modes become dispersed in a multi-mode fiber, but this is not the case in a single mode fiber, there's only one mode so there's nothing to be dispersed. Therefore, if in some situation mode dispersion is a big source of error, it's best to switch to a single mode fiber. We saw that absorption and scattering can be reduced by improving the manufacturing process because the main source is the impurities in the fiber. Bending loss, of course, is  reduced by not bending your fiber unless you absolutely have to, and even then be very mindful how much you bend it. And finally, coupling errors can be eliminated, at least partially, by ensuring that the fibers are aligned properly and there is no gap between them. But even if we try to do all of these things, there will be still some attenuation and some losses, so let's go back to our expression for how much power we lose for some input power over a distance $L$ with some attenuation parameter $\alpha$.

And always, this $\alpha$, no matter how careful we are about the manufacturing and using single mode fibers and coupling the fibers together in a proper way, we will always have the attenuation parameter $\alpha$ be non-negative. Therefore, there will be some attenuation on the signal. And in the previous step, we saw how much the signal becomes attenuated over a short distance of five kilometers, if the $\alpha$ parameter is set to twenty, or something very low like 0.18. So it seems that over the distance of five kilometers, the signal becomes attenuated, but not by that much. Still around eighty percent of the signal gets through, but five kilometers is not a very long distance. Let's try to extend this distance and see what happens. So let's extend it to something like a hundred kilometers, and immediately you see that even in a low loss fiber that has $\alpha$ parameter of 0.18, eventually the signal will go to nearly zero. After twenty kilometers, it's around forty four percent of the initial power. After only fifty kilometers, it drops to thirteen percent, and at the distance of a hundred kilometers it is virtually zero. And hundred kilometers is not such a long distance, we need systems that can reach thousands of kilometers. So the question is: how is it done? How can we use fiber optics to actually transmit signals over such extreme long distances?

And that's done with the help of something called "repeaters", and they are devices which are used to boost the signal strength. And there are many different repeaters based on many different physical principles. The ones that we are going to talk about in this lesson are called erbium-doped fiber amplifiers\index{amplifier}, or EDFA for short. How this works is that you introduce erbium atoms into the fiber. Then you pump these atoms, meaning that we excite these atoms into their higher energy state creating population inversion. Then as the weak signal that we are aiming to boost goes in, it can stimulate emission from these erbium-doped fibers- from this erbium atoms, and we know that in the process of stimulated emission, the photons that are emitted are basically of the same kind as the incoming signal photons. They have the same phase, same coherence, and they travel, most importantly, in the same direction. So basically, this is using the principle that's behind lasing to boost our weak signal. And in this process, we obtained a signal that becomes amplified and can travel further distance before it needs amplification again.

And just to remind you, we will encounter this term "repeater" in the next lesson as well, where we will be talking about quantum repeaters, but they work on a very different basis as classical repeaters, here we are still talking about classical repeaters.

So this is the image that we have in mind (see slide). We've got our fiber over here, and we've got some portion of the fiber where we introduced erbium atoms into it. We are pumping them strongly such that we create population inversion, and as the weak signal comes in, it stimulates emission from this red part where we introduce the erbium atoms, and in that process becomes boosted again, and we allow it to travel further for another fifty kilometers or so before it needs boosting again.

But, there is one problem with this approach that we have to keep in mind, and that is that EDFAs not only amplify the signal, they also amplify the noise as well. Let's see how that works.

Some of the erbium atoms, they will actually decay via spontaneous emission. Remember, we are exciting them, and they are not just waiting around for the signal to come in and then cause stimulated emission. Sometimes they get a little bit impatient and they decay spontaneously. Photons that originating from spontaneous emission are considered noise photons. They're not carrying information about our original signal. But as these photons are traveling down the fiber, they can travel either backwards, which is not really a problem for us, but they can also travel forwards, and then as they do that, they can become amplified via stimulated emission. Remember, there's a lot of erbium atoms sitting around in the excited state, just waiting for some other photon to come in and cause stimulated emission. So, what's happening in the EDFA is signal amplification, where the signal photon, so the photon that we want to amplify, comes in stimulates emission from an excited erbium atom, and in the process becomes amplified. Then, we get two signal photons, which is good, which is what we want. But, as we said, some of the erbium atoms decay spontaneously. So this is how noise amplification occurs. We have an excited erbium atom, it decays spontaneously into its ground state giving out one noise photon via spontaneous emission, and that photon can then be amplified via stimulated emission from other excited erbium atoms, producing two noise photons. So what's very important is to consider the ratio of the amplified signal to the amplified noise.

So, this is the basic principle of a classical repeater. Now we will see how the sources of noise affect the propagation of signals over long distances in the quantum case, and mainly, we will see that the set of challenges that are facing us there are completely different.



\section{Quantum challenges}

So in quantum communication, the signal is at a level of individual photons, which is very different to classical communication, where we are sending many, many photons. So the major problem becomes photon loss in the fiber. Classically, if we are communicating and we lose a single photon, this is very insignificant. The whole signal propagates and we can still read it out at the output. However, when this happens in a quantum communication protocol and we lose a single photon, then the entire protocol fails. Go back to, for example E91 or BB84, when a loss of a single photon becomes a huge problem.

So, how can we combat photon loss? We saw that it's possible classically. Let's think whether this is also possible in the quantum communication.

So amplification of a signal worked in classical communication. Can we do that for quantum signals as well? Namely, can we create backup backup copies of the single photons such that if one gets lost, we can still use the backups in order to proceed with the protocol? We can, but only if we limit ourselves to orthogonal states. If the states that we are sending are, for example, just qubits zero and one, zero and one, and so on, fine, we can do that and we can create backup copies. But, in quantum communication all the magic happens with non-orthogonal states, and often these states are entangled with some other qubits somewhere else. Therefore, we have to be able to copy arbitrary states, and this is where we hit the roadblock that we have seen in previous lessons, which is the no cloning theorem.

Okay, so amplification will not work in quantum communication. How about just sending the photon and hoping for the best?

Let's do a very quick calculation that will demonstrate that this isn't another very good strategy. So the probability that we transmit the photon through a fiber over distance $L$, where the attenuation parameter in the fiber is $\alpha$, is given by the following expression: it's ten raised to the power of negative $\alpha$, times $L$ divided by ten. So now let's plug in some numbers to give us some intuition of what this probability is. So we will consider a long fiber of thousand kilometers, which we call long but in the context of global communication is not that long actually, and we will assume a best-case scenario where the fiber has ultra low attenuation of mere zero point one decibels per kilometer. So the probability of that we are actually successfully transmit a single photon is given by ten to the power of minus ten. So this looks like a very small number, and indeed it is, but to give you some intuition of really how small this number is, let's consider that we have a source that produces one photon every second. So every second we are sending a single photon down the fiber. How long do we need to wait in order for somebody that's at the end of this fiber a thousand kilometers away to actually successfully receive a single photon? Well, we have to wait on average 317 years. So you can see that even with ultra low fibers over moderate distances, sending a single photon down the fiber is not a very good idea.

And this is only one source of error that we have to contend with in long distance quantum communication. Other sources of errors include unitary errors such as Pauli errors, where we can randomly flip the state of our photons, or z errors where we introduce some phase to the photons, and then there is a whole bunch of non-unitary errors, such as the coherence, dephasing, relaxation, most of these errors, we don't have to deal with in classical communication.

So the situation looks quite dire. Is there any hope for long-distance quantum computation? We will see that quantum problems require quantum solutions. So,  in the next chapter where we will talk about how we can overcome these challenges.



\newpage
\begin{exercises}

\exer{Confirm the calculation of 317 years.}

\end{exercises}

