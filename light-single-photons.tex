\chapter{Coherent Light and Single Photons}

\section{Introduction}

%Hi, and welcome to Lesson Five, "Coherent Light and Single Photons". This is the first lesson in the series called "Fundamentals of Quantum Optics".

%Step One, "Introduction". 

Why do we want to encode information as optical signals? Well, for one, light is a very good carrier of information because it's fast. How fast exactly? In vacuum we denote the speed of light by c and it's given by 2.997 times 10 to the 8th meters per second. However, we don't send light through vacuum. When we communicate, we can send it through air, where the light slows down by a factor of 1.0003 so really that doesn't limit the speed of light that much and it's still given by the same number virtually. Or, we can send light through optical fibers. These are pieces of glass where the speed of light  decreases by a factor of 1.467, which results in 2.045 times 10 to the 8 meters per second, so that's still very very fast. Apart from being fast, light is also relatively easy to produce. In the early days, if you wanted light you could just light the fire. Nowadays, of course, we just use lasers and send those laser signals down optical fibers. And also light does not interact with other things that easily, and therefore it is robust to noise. For example, if you compare it with copper wires where each copper wire is carrying electrons, electric signals, these create electric fields and magnetic fields that affect other wires which are carrying some other signals and therefore introduce noise, whereas this is not the case in optical fibers and you can generally pack these fibers very compactly to one in one each other and without affecting any of the messages.

So, optics has always played an important role in communication. We saw a couple of examples of that already in the Great Wall of China and Napoleon's semaphore where we are basically sending optical signals to communicate some things. However, these methods were limited in the sense that you had to have a direct visual path between the sender and the receiver and also you needed good weather conditions and they usually worked only in the daylight. Then we learned how to control light and actually guide it through waveguides known as optical fibers and all of these previous requirements disappeared and this sparked an expansion in how quickly and how far we can communicate. This is a map of the main optical fibers that go across the seas and oceans, so submarine optical fibers, and you can see how many there are connecting all the continents, basically allowing us to communicate from one side of the world to the other within a matter of milliseconds.

So in this lesson, we're going to be concerned with how to produce light and in particular we will look at three types of light. We will begin with incoherent light. So, this is light that can be produced by burning fuel or heating gas and it's called incoherent because it doesn't have any coherence in it and we will explain exactly what this means. It's very easy to produce, which is why it has played a very important historical role as we have discussed already. This type of light is known as a classical state of light, so it doesn't manifest any quantum behavior. We will then move on to coherent light produced by lasers and the main mechanism behind producing this light is known as stimulated emission, which we will discuss in the next couple of steps. This light is coherent. It's also relatively easy to produce, which sparked the first information revolution, but again it's still just a classical state of light and we will explain exactly why that is. And lastly, we will conclude this lesson by looking at single photon sources. We will look at three main ways of producing single photons: by attenuating laser light; we will consider another scheme known as heralded photons; and then also we will look at a particular physical system known as nitrogen vacancy centers in diamond, which are among many other things good sources of single photons. Compared to the previous two types of light, single photons are very difficult to make. They can be only made under very, very stringent requirements in laboratories, but they can display quantum behavior which is why they are crucial in quantum communication.

\section{Coherent vs incoherent light}

%Step two: coherent versus incoherent light. 

How does matter radiate light? What are the sources of light? Let's consider a model of a simple two-level atom. We've got our ground state with energy Eg, and we've got an excited state with energy Ee, and this green circle denotes in which state the atom is found. So currently it's in the ground state. In order to make light, we want to excite the atom to its higher state. We can do this thermally, so we give the atom thermal energy, or we can do also do it with another photon, and if the energy of the photon is tuned just right so that it is equal to the energy difference between the energy of the excited state and the energy of the ground state, then the atom will become excited and move to a state of higher energy. When it's there we'll leave it alone for a while, and at random time it will spontaneously de-excite, jump back into the ground state, releasing its energy in the form of a photon. And the energy of this photon will be given by the energy difference between the excited state and the energy of the ground state. This process is known as spontaneous emission. As I said, it is spontaneous -- it happens at random time and we don't have much control exactly when it happens. 
Now let's consider two such excited atoms. They're both sitting in their excited states, Ee and Ee, and if we wait some time one of them de-excites into its ground state and so does the other one and while this happens both of them give out light as we said, and this light is traveling in some random directions and also in some random phases. So the two photons that now we have two sources of light. The phases between these photons are different and so are the directions of travel. When this happens, we call such light incoherent light. Now you can ask yourself, what happens if we have many of these photons? Well, basically what you have is you've got a light bulb. So you've got some gas inside the light bulb and you've got a filament. If you turn on the switch, the electricity runs through the filament, it heats it up and starts giving off light. All those atoms inside the filament, they become excited, but this time the filament the atoms in the filament don't just have a simple two level structure. There are many different levels and as you give it thermal energy, atoms, they become excited to higher levels, some to slightly lower levels. So when they spontaneously emit, they give out light of different frequencies and again this light is traveling in all the possible directions and because the emissions happen at different times the light has different phases. So to conclude, incoherent light has different frequencies, it has different directions in which it is traveling, and it is out of phase so it has different phases. On the other hand, coherent light is the exact opposite of incoherent light. It is monochromatic, which means it only has a component of one frequency, it travels in the same direction, and as travels it is in phase.

So the question now is what device outputs such a light? We saw a light bulb is a source of incoherent light, so what is the source of  coherent light? We're going to find it out in the next two steps.

\section{Lasers I}

%Step three: lasers, one.

In the previous step, we said that we are looking for the source of coherent light. To remind you, coherent light is in phase, it's monochromatic, and it travels in the same direction. So you may have guessed that this source is a laser, but what does laser stand for? It's an acronym for "light amplification by stimulated emission of radiation", so let's break down this title a little bit and see what it means. First, we will talk about stimulated emission. This is the physical process behind lasing. It's a new type of light-matter interaction, and it is responsible for producing coherent light. We will also see that in the process of producing this coherent light, the light itself gets amplified. This allows us to create very intense light, much more intense when compared to incandescent light sources. So let's review the three fundamental light matter interactions. We have already seen two of those. The first one, let's consider our again our two level atom which is in the excited state. After some time, we said that such an atom will de-excite. It will transition into its ground state and in the process of doing that it will give out energy in the form of a photon traveling in some random direction. This is a spontaneous emission. We have also seen how to excite an atom.

By having it interact with a photon tuned to the energy difference between the energies of the excited state and the ground state, such a photon can interact with an atom in its ground state. The atom then receives this energy, becomes excited and transitions to the excited state. This is known as stimulated absorption. The third fundamental light matter interaction is the following: we have the atom. It is sitting in the excited state, and this time a photon of the right frequency corresponding to the energy difference between the excited state and the ground state comes along and it stimulates the atom, and what can happen is that the atom transitions down to its ground state without absorbing the initial photon, of course. So the initial photon is there, but while the atom is transitioning from the excited state to the ground state it also has to give out energy and it does it in a form of another photon and this process is known as stimulated emission.

And these two photons, they have the same frequencies, they travel in the same direction, they are of the same phase, and on top of that they also have a same polarization. So these two photons are coherent with each other. Therefore stimulated emission produces coherent light.

And as you may have seen or understood from the previous slide, such a light also gets amplified. Why is that? Imagine that this green box represents our excited atom. So we have one excited atom that interacts with some incoming photon. And we saw that this scenario produces two photons. Now these two photons can separately interact with two more excited atoms. So after stimulated emission what we get is we've got four photons. So you see we've got this cascading effect where excited atoms are producing more and more photons which are coherent. However, there is one problem, and that's that not all of the atoms are usually found in the excited state. Getting an atom into an excited state is actually quite a difficult process. Getting all of them into an excited state is way more difficult. So let's do some basic accounting. Let's say that we've got an atom and then we've got one incoming photon. What can happen is that this photon completely ignores the atom, nothing happens. So we've got one photon in and we've got one photon out. Then we've got the other scenario where the atom is found in the ground state and it interacts with an incoming photon. And what happens through stimulated absorption the photon gets absorbed, atom gets excited, so there are no photons coming out. Or, we saw what can happen through the process of stimulated emission. If the atom is in the excited state the incoming photon stimulates the atom. It jumps, it transitions to the ground state, and we obtain two photons coming out. So let's look at the first first one. When no interaction occurs we've got one photon coming in, one photon coming out. The number of photons is conserved. If there is some interaction, like in these two cases, we've got two photons coming in and also we have two photons coming out, so again number of photons has not really changed.

Let's consider the following scenario. Let's start with all the atoms (in this case we have four of them) in the ground state. This capital Ng represents the number of photons [should be atoms] in the ground state and capital Ne represents the number of photons [should be atoms] in the excited state. And let's say that we've got some photon coming along. What can happen is that only stimulated absorption because all of the atoms are found in the ground state. So let's say that the second one becomes excited. It absorbs the photon and transitions to the excited state. So fine, now we've got three atoms in the ground state and one atom in the excited state. Then another photon comes along, again of the right frequency, and what can happen? Well, now we have two possibilities. One possibility is that the atom [should be photon] gets absorbed and excites one of the three remaining atoms, or it interacts with that one atom in the excited state and causes stimulated emission. But there's only one atom in the excited state and there are three atoms in the ground state, so it's far more likely that stimulated absorption occurs. So let's just say for the sake of concreteness that this is the case. So again stimulated absorption is more likely to occur so and let's say that the first atom gets excited. Now we've got equal number of atoms in the ground state as atoms in the excited state. So let's go on and consider another photon coming along. Now the probability that this photon will become absorbed or that it stimulates emission from one of the excited atoms is equal, because two atoms are in the ground state, two atoms are in the excited state. But let's say that this particular photon becomes absorbed. So what happens? One of the photons [should be atoms] from the ground state transitions into the excited state. And now you see that if another photon comes along, it is far more likely that actually it stimulates an emission from one of these three atoms in the excited state, because there's more of them. So this shows us that in order to have a good chance of stimulated emission, we require most of the atoms, or the majority of the atoms, to be in the excited state. So we write it down as follows. We require that Ne is larger than Ng. And this condition this situation is referred to as population inversion, and that's because we inverted the population from the ground state, where majority is in the excited state

However, there's still one more obstacle. How do we actually achieve population inversion? Because consider this: if the majority of our atoms are in the ground state then it's far more likely to absorb the incoming photon so the Ne, the number of excited atoms, increases. On the other hand, if we already have population inversion, if we have the majority of the atoms in the excited state, then we are more likely to cause stimulated emission, which increases the number of photons in the ground state, meaning we are losing the atoms contributing towards the population inversion. So what tends to happen in a two-level system is that the system likes to pick some type of equilibrium where the number of excited states and the number of atoms in the ground state are approximately equal. So this is not exactly what we want. We are aiming to maintain our atoms in the excited state. We want population inversion. So how do we do it? And the answer to that we will find lies in abandoning our two level atomic friends and moving at least to a three-level atom. And we're going to explain exactly how that works in the next step.

\section{Lasers II}

%Step Four: Lasers - Two

We said in the previous step that in order to create "population inversion", we require a three level atom. So, this is our new atom. It's got three levels and we rename the ground state to E1, the previous excited state was E, but now we're going to call it E2, and then there's another level of higher energy which we are going to call E3.

And, this new introduced level is an unstable level, meaning that whenever we excite the atom to energy E3, then it quickly decays- it doesn't spend much time in this highest level, and this transition between E3 and E1 will be referred to as the "pumping transition".

And this transition between E2 and E1 is our original "lasing transition".

So the goal is to create population inversion by exciting the atom to E2, but without actually affecting the lasing transition, meaning somehow we need to excite the atom to E2 without causing stimulated absorption between E1 and E2. So let's see how to do that. Consider a photon coming in that's tuned to the pumping transition's energy, so its energy is equal to the difference between E3 and E1. What that does it causes stimulated absorption of the ground state atom. It absorbs the energy of the photon and it gets excited to energy level E3. And at the same time, it does not affect the lasing transition at all, so the atom does not get excited from E1 to E2 purely because the photon- the blue photon that caused the stimulated absorption is tuned to the transition between E3 and E1. And as we said, the level E3 is unstable, so it quickly decays to level E2, and then once it's there it can interact with an incoming photon that's of the correct frequency given by the lasing transition, and as we saw what can happen is stimulated emission, and we obtain two photons of the same frequency, traveling in the same direction, and that are both in phase meaning that they are coherent. So this is the cycle. We are irradiating the photon by two light sources. One light source- this black one, is tuned to the lasing transition and it's responsible for causing the stimulated emission, and that's basically our laser light. And there is also this pumping photon coming in that's responsible for exciting the atom to E3 which then quickly decays to E2. So the pumping photon is responsible for creating a population inversion.

%And that's what we said.

This is what we predicted. \rdv{? okay ?}

What we end up with is if we have many such three-level atoms, we end up in the situation where N2- so the number of atoms in the level E2, is larger than the number of atoms in level E1, and if the pumping rate is high enough then all of the atoms that are in the ground state, they are more likely to get pumped into E2 rather than absorbing the incoming black photons and getting excited to level E2.

So, now let's start to build our laser. All of these dots- they represent our three level atoms, and we're going to refer to this as a "gain medium". And a gain medium can be either solid, it can be a gas, or it can be a liquid. So without doing anything, most of the atoms are found in the ground state. Occasionally, some of them get excited and what happens is they decay via spontaneous emission, and they are giving out light in all directions. But, we want to obtain lasing action- we want all of the light to be coherent. So what we do- as we saw, we must create population inversion, and to do that we start pumping the medium represented by these blue arrows, so what that does it excites our atoms first to level E3, then it quickly decays to level E2.

We, in fact, do get some stimulated emission, but still we are getting also some spontaneous emission, but most importantly the photons that we get via stimulated emission- they are escaping our gain medium. What we can do to prevent this is actually build mirrors from both sides. This will increase the number of photons inside the gain medium that will later be part of this cascading process of creating more and more coherent photons. So now we are pumping our gain medium, we've got 100 percent reflective mirrors on both sides, and what happens, in fact, we see lasing action inside the medium. The pumping creates population inversion and then the existing photons that are inside the gain medium stimulate emission on the lasing transition, creating more copies of those photons and again cascading into a lasing action.

So, we created a laser. The only problem is- the light is confined within the gain medium, and somehow we need to get it out. So what we do- we take one of these mirrors and we make it partially reflective, and "partially" here means that it's 99 percent reflective. So what happens, the photons inside the gain medium, they've got some probability to actually escape out and they in fact do, and what we get is a nice, monochromatic, in-phase, traveling-in-the-same-direction light, and this is basically the basic principle of constructing a laser.

And as we said, the output is coherent.

So, we said that we have to pump our three-level atoms in the gain medium at some level, we have to pump it hard enough. So can we get some physical understanding where the threshold is? And for that, we will write down a very simple, a little bit naive model of how our laser works. Let's say that the number of photons in the gain medium is given by small "n", and that's a function of time (t), the number of excited atoms that we care about because they contribute towards the stimulated emission process, is denoted by capital "N", and that's also a function of time (t).

And the rate of change of photons inside the gain medium is given by the difference between some gain process and some loss process. So let's examine what's the gain. Well, the gain depends both, on the number of photons that are already in the gain medium, and on the number of excited atoms. If we don't have any photons, then we cannot cause any stimulated emission. If we don't have any excited atoms, again we don't get any stimulated emission. So we can model the gain term as follows- we've got some "G"- some proportionality constant, which we're just going to call the "gain coefficient", times the number of photons, times the number of excited atoms. Okay, how about the loss? Well as you saw, we lose photons by these photons escaping through the partially reflective mirror. So we can just model it simply as some coefficient "k", which we're going to refer as the "loss coefficient", times the number of photons that are already in the gain medium. If we don't have any photons we have nothing to lose. If we have more photons we are more likely to lose them.

So, this is our model, that the rate of change of the photons inside the gain medium is given by these two terms- gain minus loss. And now comes the crucial bit- this capital N, the number of excited atoms, also depends on time, mainly the stimulated emission that happens decreases the number of excited atoms. And let's say that our pump has the ability to keep the number of excited atoms at some constant level "N-zero" if there is no lasing action, so if there is no stimulated emission happening, this is the number of excited atoms that we have due to the pumping process.

So we can write down an equation for the number of excited atoms simply as: N-zero minus alpha-n, where this "alpha" is the rate of stimulated emission. So substituting everything back into the rate equation for the number of photons, we get the following non-linear dynamical process.

We've got this linear term "n", which depends depends on some combination of these parameters, and then we've got a non-linear term n-squared, which depends on alpha times g. And as you can see, all of our parameters are positive, therefore the graph of this "n-dot" is always a concave parabola. So let's consider one case and plot this graph. On this vertical axis we've got the rate of change, and on the horizontal axis we've got number of photons inside the gain medium. So if n-dot is positive, it means we are increasing the number of photons in the gain medium, and this happens when we have lasing. However, if n-dot is negative, it means that the number of photons is decreasing, there is no lasing, we are simply losing photons, and you can see that if N-zero is such that it is less than k over G, this first linear term in our n-dot equation is negative, and then what happens is our n-dot looks looks like that, and the little arrow on the horizontal axis means that it doesn't matter with how many photons you start in your gain medium. You will eventually lose all of them, meaning the fixed point of this process is at n equals to zero, so no matter how many photons you start, at the end of the day you will always lose your photons. And that's because the coefficient in front of the linear term is negative, but of course we can increase zero, we can just increase the pumping strength such that the coefficient of the linear term is positive, and that's the following scenario. And you see that now, we've got two fixed points- one is still at n equals to zero, but this time this fixed point is unstable, meaning if we have zero photons in the gain medium to start with, of course we will not see any lasing action, and we will remain at n equals zero. However if we just slightly perturb from this situation, if we increase the photon number just slightly, n-dot in that region is positive. It means that we are getting more photons, so we will move along the n-axis until we reach this black circle which is our new stable fixed point. On the other hand, if we start on the right of this fixed point, we start with more photons. We will lose some of them, but we will stop at this finite fixed point, meaning that in this regime when N-zero is larger than the ratio of k and G, we get lazing action.

And we can see that if we plot the number of photons as a function of the pumping strength, which in our model corresponds to capital N-zero.

And in the region where N-zero is less than k over G, our fixed point in terms of n is just zero. Always, we lose all of the photons from our gain medium. But once we increase the pumping strength past this threshold given by k over G, we see that the fixed point for n keeps increasing and is finite. So, in the region where N-zero is less than k over G, we basically have a light bulb. The only source of light that comes from our gain medium is incoherent light. On the other hand, if the pumping is strong enough and N-zero is larger than k over G, we have a laser.

\section{Single photons}

%Step Five: Single Photons

So we have seen how to generate light in an incoherent state. We have seen how to generate light in the coherence state. And satellite is very useful in classical communication. In particular, it uses- in order to encode zeros and ones in order to code classical bits, it uses large numbers of photons. So on a on a graph, where on the horizontal axis we've got time, and the number of photons on the vertical axis, all of these ones and zeros- these classical bits, here they are represented by some very, very large number. And this has a number of advantages. Number one- it's robust. Even if you lose some of the photons, it's no real big deal. Of course, eventually you will lose most of your photons and your signal will deteriorate, so then what you can do, you can just easily amplify and boost the signal back to its original strength. And also, it allows for quick and deterministic generation. However, such signals are not very useful because we cannot put them in a quantum superposition, and on top of that, because we also cannot entangle such signals between each other, and therefore it's not very useful for quantum communication. Our signals, in order to truly take advantage of quantum mechanics, have to be quantum as well, and for that we have to go beyond lasers and look at single photon sources.

One way of generating single photons is by using attenuated light.

Consider the following scenario- you've got your laser source, and then you've got a bunch of attenuator plates over here represented by these bands. You turn on your laser, it starts at some intensity, but as it passes through the attenuators it loses its intensity until it's so faint that you truly get only single photons represented by these discrete dots over here after the last attenuator. And in principle, this works and we will use this in the next lesson when we look at interference between single photons. However, one big problem of this is that the arrival of these photons is random- it's given by Poissonian statistics, and in order to really be useful in quantum communication, we need something that's not random. So for the purpose of quantum communication, such a scheme is not very valuable and completely not practical. A different scheme that yields single photons is using some kind of heralding mechanism. For example, you can produce two photons, detect one which will also herald the presence of the other, and we already saw such a process in previous lessons, and to remind you it's called "spontaneous parametric down-conversion" (SPDC). So here, you've got some nonlinear crystal, you shine light of certain frequency at the crystal and that produces two photons as your output. Before we used it to produce entangled photons- they were entangled in their polarization. However, here what we can do, we can just detect one of the photons, and if we do then we know that the other photon is also present.

And a couple of advantages of this heralded scheme is that these photons- they travel along a predetermined path, meaning if we detect a photon by our detector, if the detector goes click, then we know exactly where the other photon is traveling. Not only its presence, but also in which direction it's traveling. So it's very easy to couple it to some fiber and then send it off across our quantum network. And on top of that, the rarity of SPDC is actually an advantage. If you recall from previous lessons, this parametric down-conversion, this process, is actually very rare. At best, only we get one pair out of a million photons in state-of-the-art experiments. However here, in this context, that's actually a good thing. It guarantees that if our detector goes click, we've got very high probability that we truly have a single photon and not two or more photons. If SPDC process was more efficient, then it could happen that our detector goes click and here we've got just too many photons, which is not useful for quantum communication.

However, this whole process is still random. Sometimes we get this pair of photons, and even if we get a pair of photons, we have to consider the inefficiencies in the detector. Sometimes it goes click when there isn't a photon. These are known as "dark counts". Sometimes there is a photon and the detector doesn't detect it. So what we are looking for is a single photon, nearly-deterministic source. And such schemes actually do exist.

In particular, you can consider the following structure- again, it's a three-level structure with levels E1, E2, E3, and what you can do is you can excite the atom to level three, which again is short-lived as we saw in our discussion of lasers. It quickly decays to a long-lived level two, and in particular, this level two is long-lived enough such that the pulse that is used to excite to level E3 is finished. So there's a very clear separation between when the exciting pulse is finished and when the atom de-excites from level two to level one, emitting a photon, and it's this photon that is nearly deterministic due to the stability of level two.

Do such materials exist that have this property? In fact, yes, and one example of such a material is known as nitrogen vacancy (NV) centers in diamond. This is a particular material that has many, many nice properties which are very useful in quantum information processing and in quantum communication, and we will come across it again and again. Basically, what it is is- you've got a crystal structure. These green dots- they represent carbon, and you knock one of the carbons out and you replace it with a nitrogen and you knock the other one, and you create a vacancy.

And in particular, these systems have very long coherence times, so we will see them again when we talk about quantum communication protocols, and also quantum memories.


\newpage
\begin{exercises}
\exer{Consider the following quantum state:}
\begin{equation*}
\ket{\psi} = \frac{\sqrt{3}}{2}\ket{0} + \frac{1}{2}\ket{1}
\end{equation*}
\subexer{Find the probability of measuring a zero.}
\subexer{Find the probability of measuring a one.}


\end{exercises}

