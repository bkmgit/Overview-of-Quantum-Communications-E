\chapter*{Using This Book}

\section*{In a Degree Curriculum}

This book represents about eleven hours of online video~\footnote{\url{https://www.youtube.com/playlist?list=PLCTGenrx1-SOC-b98RCC1uEGI-Sc-N3C-} or find the corresponding playlist below \url{https://www.youtube.com/c/QuantumCommEdu}.}, organized as a set of fifteen lessons corresponding to the chapters of this book. Coupled with homework assignments, it could be twenty to thirty hours of work for well-prepared students (and substantially more for not-as-well-prepared students who build their understanding along the way).  As such, it is intended to represent one credit toward graduation by Japanese standards, where an undergraduate degree is typically 124 credits over eight 15-week semesters. As typical courses are two credits, this content is expected to be paired with additional material in a Japanese curriculum to form a single for-credit course. In the United States or elsewhere where the typical course is a larger chunk of learning, this material may form a third or a quarter of a course.

In conjunction with the videos and this book, we recommend using the material in an active learning context, such as a \emph{flipped classroom} environment.

Additional details, such as the recommended mathematics background, are covered in Sec.~\ref{sec:mod-over}.

\section*{In a Tutorial}

It is also possible to take a subset of this material and use it in a half-day or full-day tutorial, with appropriate background preparation.  For example, at the Third Workshop for Quantum Repeaters and Networks, held in Chicago in August 2022, we asked attendees to prepare by watching the following subset of the videos plus part of a seminar given by Prof. Van Meter:
\begin{itemize}
\item QSI Seminar: Prof. Rodney Van Meter, Keio University, Engineering the Quantum Internet, 30/06/2020~\footnote{\url{https://www.youtube.com/watch?v=FZsEOhM4mIY}}
At minimum, watch from 9:45 to 27:30 on 
“Applications of a Quantum Internet”
\item 2 Quantum States
\begin{itemize}
\item 2-1 Qubits
\item 2-2 Unitary operations				
\item 2-3 Measurement
\item 2-4 Probabilities, expectation, variance
\item 2-5 Multiple qubits
\end{itemize}
\item 3 Pure and Mixed States
\begin{itemize}
\item 3-1 Noisy world
\item 3-2 Outer product
\item 3-3 Density matrices
\item 3-4 Pure vs mixed states
\item 3-5 Fidelity
\end{itemize}
\item 4 Entanglement
\begin{itemize}
\item 4-1 CHSH Game
\item 4-2 Entangled states
\item 4-3 Bell states
\item 4-4 SPDC
\item 4-5 Entanglement as a resource
\end{itemize}
\item 8 Teleportation
\begin{itemize}
\item 8-1 Introduction to teleportation
\item 8-2 Teleportation protocol
\item 8-3 No-cloning theorem and FTL communication
\end{itemize}
\item 12 Quantum Repeaters
\begin{itemize}
\item 12-1 The need for repeaters
\item 12-2 Making link-level entanglement
\item 12-3 Reaching for distance: Entanglement swapping
\item 12-4 Detecting errors: purification
\item 12-5 Making a network 
\end{itemize}
\end{itemize}

At the workshop, we conducted a four-hour session consisting primarily of hands-on work.  The hands-on work demonstrated the three key concepts of \emph{teleportation}, \emph{entanglement swapping}, and \emph{purification}. The Qiskit Jupyter notebook~\footnote{\url{https://github.com/sfc-aqua/wqrn-tutorial}} and QuISP (Quantum Internet Simulation Package)~\footnote{\url{https://github.com/sfc-aqua/quisp}} demos used in the hands-on session are also available on the web. \rdv{Add actual Jupyter text/code as an appendix?}

\section*{Reusing, Recompiling, Reorganizing and Contributing to the Material}

You are welcome to recompile this book to meet your own local needs, adding, removing or rearranging material as you please. For example, if you use this book in a graduate-level course or after students have a substantial background in quantum information, the introductory material in Chapters 2 and 3 may be unnecessary. Or, you may wish to combine chapters from this book with other material of your own to create a single volume of lecture notes for a course at your institution.

The Creative Commons wiki provides some additional explanation on the license~\footnote{\url{https://wiki.creativecommons.org/wiki/ShareAlike_interpretation}.}.  The source for the book is available on GitHub~\footnote{\url{https://github.com/sfc-aqua/Overview-of-Quantum-Communications-E}}. 

Contributions to the book itself are welcome.  The text can always be improved; it began life as direct transcripts of the videos, and as such does not always flow as smoothly in book form.  Contributions of exercises for students are especially welcome.  Additions that substantially modify the flow of material may be organized as supplementary material at the end of chapters or the book, in order to maintain the correspondence between the online videos and the text.  The preferred form of contributions is as \emph{pull requests} on Github.  Those whose contributions are accepted will have their names and affiliations added to the contributors page, at their discretion.
