\chapter{E91: Entanglement-based QKD}

%Hi, and w
Welcome to Lesson Ten on Entanglement-based Quantum Key Distribution (QKD), and in particular, an entanglement-based protocol known as E91.

\section{Introduction}

% Step One: Introduction

Recall that in the previous lesson, we learned about the BB84 single photon-based quantum key distribution protocol. And to remind you how it works, let's do a little overview. We have Alice and Bob that use a public quantum channel to establish a secret key. What Alice does is she prepares qubits in four different states at random. The states can be either zero, one, or they can be plus, or minus, and then she transmits these states to Bob. Then what Bob does, is he randomly measures them either in X or Z basis. After the measurement is finished, Alice and Bob exchange information about the preparation bases and the measurement bases, and if these coincide, they keep results for those measurements, and that forms the basis for their secret key. And if they dedicate a portion of the security key to eavesdropper detection, they can do that due to non-orthogonality of the original encoded qubit states. Now, imagine that there's an eavesdropper, Eve, and we said that she cannot know the preparation basis. We will demonstrate why. So, consider the case that she does have information about the bases in which Alice prepared the original qubits. What then she can do is she intercepts the first qubit, and she knows that this qubit was prepared in the Z basis, therefore she measures it in the Z basis and obtains the corresponding classical bit which in this case is zero, and then she just resends the photon back to Bob. She intercepts the second qubit, and again because she knows the information about Alice's preparation basis, she measures in the appropriate basis, which in this case is the X basis, and because the case is minus, she obtains a classical bit one. And she repeats this procedure so on and so forth. She measures in the Pauli Z basis for the third qubit, she obtains a classical bit zero, and then resends that qubit back to Bob. So what she is doing, is although, she's measuring these qubits, she is not disturbing them at all because she's always measuring in the same bases in which they were prepared. So in this way, she can actually build up a secret key that's perfectly correlated with the key that Alice and Bob are sharing.

Of course, this is a big problem because then the whole procedure of BB84 fails. Even though Alice and Bob, they can try and detect Eve as they would in the normal protocol, but she has not disturbed any of the qubits. Therefore they will never detect her presence. So in this lesson, we will go over a protocol that's a little bit more secure in this sense, and it relies on a pre-shared entanglement between Alice and Bob. We will assume that Alice and Bob can communicate over a classical channel, and also that there is some source of entangled states, and this source generates multiple copies of an entangled state and distributes it to Alice and to Bob. However, we will see that in this protocol- an entanglement-based protocol, even if the source of the entangled qubit- so the main resource in this protocol is Eve, the protocol still remains secure in the sense that Alice and Bob can easily detect an eavesdropping Eve. So, let's learn about this protocol.

\section{Basic ingredients}
%Step Two: Basic Ingredients

There are two basic ingredients to 
our entanglement-based QKD protocol.
The first ingredient is the procedure of 
establishing a secret key. So we said that
for that purpose, we will use an entangled state 
of two qubits. So let's consider the case where
Alice and Bob are sharing the following Bell 
pair. The state is described by this state,
vector psi plus, which is an equal superposition 
of basis states zero one and one zero.

Now, if we measure these qubits in 
the same basis, the outcomes will be
correlated or anti-correlated depending in which 
basis they are measured. And also, the probability
of these outcomes is uniformly random. So let's 
see an example to demonstrate how it works.

Let's say that- as we said we start in this 
Bell state psi plus, and both Alice and Bob
measure in the X basis. Then we can compute the 
probabilities of all four possible outcomes.
So the probability that both Alice and Bob 
obtain a correlated result of plus plus,
is given by a half. The probability that they 
get an outcome minus minus, is also a half,
and the other two probabilities 
therefore have to be zero.

In this way, when Alice measures state 
plus, Bob always measures state plus. When
Alice measures state minus, Bob always measures 
state minus. So in terms of their classical bits
that they get as the outcomes of their 
measurements, they always get either zero
zero with probability a half, or one one with the 
same probability of fifty percent. So in this way,
they can establish a secret random correlated 
key.

What if they measure in the Z basis? Well,
the scenario is very similar, although 
now the results are anti-correlated.
So the probability that they get 
both of them get state zero zero
is zero, same for the probability of 
obtaining the state one one which is zero.
And with equal probability they either get 
the state zero one or one zero. So always,
when Alice measures zero and she's measuring in 
the Z basis and Bob also measures in the Z basis,
she knows that he always gets the state one, and 
vice versa, if she measures the state one, Bob
will always measure state zero. So in this case, 
the classical key that they get is anti-correlated
as well, just like the quantum results. So 
Alice either has zero and Bob has one with
probability fifty percent, or it's vice versa, 
Alice has one and Bob has zero. Well, but, they
know that they are both measuring the Z basis, 
so all that Bob needs to do is flip his bit and
they again establish a correlated secret random 
key which they can use to encrypt their data.

The second ingredient now is to verify that they 
have an entangled state. Why do they need to do
that? Well, the first reason is we have just seen 
that entangled states can be used to generate a
correlated random key. But also, there is a very 
important second step which was not present in
the BB84 protocol, and that is that entanglement 
can be used for security as well, namely maximally
entangled states are guaranteed to be secure due 
to something known as "monogamy of entanglement".

So what is monogamy of entanglement?
Monogamy of entanglement is a very 
fundamental property of quantum states,
and it constrains how correlated
multiple qubits can be. In particular, if Alice 
and Bob share a maximally entangled state,
then we are guaranteed that they cannot share 
any correlations with a third party, such as Eve.

In terms of security this is very important 
because if they can demonstrate and verify that
they have a maximally entangled state, they 
are automatically demonstrating that whatever
key they establish is secure and Eve does not 
have any information about their secret key.

In general, there is a trade-off. So, if 
Alice and Bob share some entanglement,
they will share some correlations with Eve. So 
the stronger the entanglement that they share,
the less correlated they are with Eve, 
until, well, to the point where they
are maximally entangled and therefore 
they share no correlations with Eve.
So, stronger entanglement between Alice and Bob- 
it implies more secure key between Alice and Bob.

So how do we actually verify that Alice and 
Bob are sharing a maximally entangled key?
We use something known as CHSH inequality.

Let's start by considering four classical random 
variables, and we're going to denote them as A,
A-bar, B, and B-bar, and they all can 
have values plus one or minus one.
And now, let's say that we form the next function 
with these random variables. We take B and B-bar,
and we add them together and multiply by the 
value of A. We also take B and B-bar and take
the difference between them and multiply by A-bar, 
and then we sum the two together. You can easily
convince yourself that for any combination of 
plus one or minus one, the maximum value that
you can get for this expression is plus two, and 
the minimum value that you can get is minus two.

Now imagine that you are constantly generating 
these random variables- A, A-bar, and you are
interested on average what expression do you get 
here. Well, that will be constraint. So here,
these angular brackets denote the average value 
of this following expression, and we are adding
it with the average value of this expression, 
and we take the absolute value of the whole sum,
and that is constrained to be less or equal to 
two. As we said above here, the maximum they can
get is two. The minimum they can get is minus two. 
So those are the two extremes, and the expectation
value will be somewhere in between depending 
on the details of the probability distributions
for this random classical variables. So we can 
just expand these expectation values over here,
and what we get is the following expression 
which we are going to denote by this "S",
and refer to it as CHSH expression, and that 
as we said has to be less or equal to two.
And this inequality is known 
as the CHSH inequality.
So any classical random variable A, A-bar, B, 
B-bar, they have to satisfy this constraint.

Okay, this is the classical case. What happens 
in the quantum case? Well, in the quantum case,
we can consider A, A-bar, B, B-bar, to be the 
measurement outcomes in a certain basis on some
state psi. So just to remind you, the expectation 
value of an observable where Alice measures A and
Bob measures B is given by this expression. Take 
the tensor product of the observables A and B,
and then we compute the following expectation 
value with respect to the state psi.

And amazingly, for some quantum states we 
can actually violate the CHSH inequality.
By violating, we mean that over here, we 
can obtain a value that's larger than two.
So then, what it means is we can use 
this expression to detect entanglement.

In particular, in an experiment when we measure 
and compute these various expectation values,
and then we sum them up in this manner and we 
obtain a CHSH expression which is less than two,
then we can say maybe the states are classically 
correlated. But, if we measure a CHSH expression
which is larger than two, then we can, in fact, 
say that definitely these states are entangled.

And in quantum mechanics, the CHSH expression 
can go all the way up to a value of two times the
square root of two. And this happens for maximally 
mixed states, as we will see in the next slide.

Let's consider a particular example. Let's 
say we take one of the Bell pairs, psi plus,
and we know that it's a maximally entangled 
state. And for the measurement settings,
we consider the following- A is the Pauli Z 
observable, A-bar is the X observable, and B
and B-bar are given by these combinations of Z and 
X, so these are just rotated measurement bases.

Then, we can just go through the algebra of 
computing the expectation values, and what we get
is, in fact, that for a maximally entangled state, 
we obtain this CHSH expression of two root two.

This gives us a way of verifying entangled 
states, and particularly verifying maximally
entangled states, which is very important for 
our entanglement-based QKD protocol. So if we can
demonstrate that we violate the CHSH inequality 
maximally, in other words the CHSH expression is
two root two, then we can certify that, in fact, 
the state that Alice and Bob share is a maximally
entangled state. That then allows us to say 
that they are not correlated with Eve due
to monogamy of entanglement, and therefore we can 
guarantee the security of their secret random key.

\section{Protocol}

% Step Three: Protocol

So we have described the two basic ingredients of E91 protocol. Now let's put them together.

So like we said, this setting is the following- Alice and Bob can communicate over a classical channel, and they share multiple copies of a maximally entangled state. And again, these copies can be generated by Eve herself.

Then, Alice and Bob randomly choose a measurement basis in which they measure their qubits. Alice chooses from these following three measurement bases. So this circle represents the exact plane of a block sphere. So her measurement setting or measurement basis A1, corresponds to measurement in the Z basis. If she does that, she projects the state either into a zero or into a one. She can also measure in the X basis, given by the horizontal direction here. Or, she can measure by a rotated basis A3, which is a linear combination of Z plus X. Bob, on the other hand, can measure also in the Z basis given by B1, or in this rotated basis B2 which is Z minus X, or in the basis B3 which is over here given as Z plus X.

Why do we have three different measurements for Alice and three different measurements for Bob rather than two like we had in the previous protocol BB84?

And this is- some of these measurements are overlapping, and this is needed for generating the secret random key. Remember, we said that if both Alice and Bob measure the entangled state in the same basis, they can use that information via are classical outcomes to generate and establish a classical correlated random key. On the other hand, we need some rotated bases such as this A3, and B2, and B3, in order to compute the CHSH expression and see if it violates the classical CHSH inequality in order to establish that Alice and Bob are really sharing an entangled state.

In order to establish the key, Alice measures either in A1 or A3, and Bob measures in B1 or B3. So, they randomly measure their multiple copies of entangled states, and then they exchange information about the basis of their measurements. So, for example, Alice has the following choices- A1, A3, A1, A2, A3, A3, A1, A3, and so on. And Bob has some other random string of measurement choices, B1, B2, B3, B1, and so on. They exchange the information about these bases and they look at the places where their measurement basis choice coincide. So in this case, it's over here. Here, Alice measures A1 and Bob measures B1, meaning both of them measured in the Z basis. If they do that, as we saw, they get anti-correlated outcomes, which they can use to generate a correlated classical key. Over here, again, they measure in the Z basis, and here they measure both in the rotated A3 B3 basis.

That takes care of generating the key. In some other cases, they will not measure in bases that coincide. But that's all right. They don't discard these results, instead they use them to compute the CHSH expression and check if they get the corresponding violation of the classical CHSH inequality. In particular, they look for scenarios where both of them measure (A1, B3), or (A1, B2), (A2, B2), or (A2, B3). So, visually it does correspond to Alice looking for cases where she measures in the Z basis and in the X basis, and Bob measures in this rotated bases B2 and B3, and then they use those measurement results to compute the following expression, which is just the sum of expectation values where Alice measured B1 and Bob measured B2, Alice measured A1 and Bob measured B3, and so on. So this way, they don't need to discard any information like in BB84, but they really get to use it to calculate either the secret correlated key or the CHSH violation. And if they obtain a CHSH expression of less or equal to two, they say "okay, we cannot conclude that we have an entangled state or not, but it's safer to just abort". Remember, we said that monogamy of entanglement ensures that if they have an entangled state, then Eve is not strongly correlated with either of them. And in particular, if they have a maximally entangled state, then Eve is not correlated with Alice and Bob at all, so they are looking for as strong a violation as they can get. And if they have a CHSH expression larger than two, then they conclude, "yes, we are sharing an entangled state, therefore we can proceed with the protocol".

So far, we have considered the case where everything was ideal. There was no noise. But what happens in the real life? How does noise affect our QKD protocols? So in real life Alice and Bob will not be able to generate a perfectly correlated key, meaning that either noise or the tinkering of Eve will introduce some inconsistencies into the key, and therefore the key will be nearly identical. Even if Eve is not trying to actively eavesdrop and disrupt the protocol, still due to inherent noise in the system, these keys will not be perfectly correlated.

What then Alice and Bob have to decide, they must decide on the acceptable security risk even if the keys are not perfectly correlated. They have to say- "okay, if the correlation is not a hundred percent, but it's very close to hundred percent, we can still use this to do something useful, and use it for a secret communication". If they do that, then they have to engage in two more protocols. One is called the "information reconciliation". That takes the initial secret key that's not perfectly correlated and produces a more correlated key. So it's increasing the correlation between Alice and Bob in their secret key. And furthermore, they also can perform something known as "privacy amplification", where they take their generated secret key and they produce a shorter key which is more secure. So they are basically trying to eliminate any possible correlation with Eve. So, which protocol is better? We have talked about BB84 and the E91. One is based on the indistinguishability of single photons prepared in non-orthogonal bases, and the other one is using entanglement-based QKD. And in particular, we care about the security of both of these protocols. So the question we can ask is- when is the secret key generated?

In the case of the BB84 protocol, it's generated when Alice generates her random string right at the beginning of the protocol. Remember, she generates two random n-bit strings. One is to encode the information about the basis of preparation, and the other one about the states in this basis. So if she chooses Z basis, is it a zero or a one, if she chooses the X basis, is it a plus or a minus. So the key- the secret key exists right from the beginning before any communication between Bob and Alice take place.

That means that a clever Eve can actually find a way how to obtain some information about this secret bit string "B". In particular, you can consider a very paranoid scenario where the random number generated that Alice is using to generate a random bit string was actually produced by Eve and is in some way correlated with Eve, therefore whatever bits- random bit string that the device produces, that information gets passed on to Eve. And we saw at the beginning of this step- of this lesson, that that poses a huge security risk for BB84 protocol. Whereas in the E91 protocol, the secret key is really generated after the entangled pairs of qubits are measured. So not when Eve produces those entangled pairs, not when they arrive to Alice and Bob, but only after Alice and Bob measure them in their random bases.

In that sense, we can say that the key is unconditionally secure, and we see that entanglement is very essential for security.

Now, let's conclude this step by talking about some entanglement-based QKD experiments. We saw in BB84 that there were network testbeds for single photon QKD networks, however the entanglement-based QKD is not as far but it only exists at the level of establishing a secret key over a single link. One such experiment was performed over free space, meaning that the entangled photons traveled through through air, and it was done over a distance of hundred and forty four kilometers between two islands in the Canary Islands, one was La Palma and the other one Tenerife. And these photons were produced by spontaneous parametric down-conversion process which we saw in previous lessons, so the entire pair of photons was encoded in the polarization of the photons. And the obtained CHSH violation, it was of 2.508, so quite a substantial amount above the classical value of two.

A different, more recent experiment was done over a distance of hundreds of kilometers, but it was done in a lab and over optical fiber. So the fiber was very long and it was wound in the lab like this, and the one distance that was tested was three hundred and eleven kilometers over standard fiber, and the other distance was four hundred and four kilometers over an ultra low loss fiber, and the obtained bitrate for the secret key was of the order of ten to the minus three, or ten to the minus four for the longer distance. Now, these bit rates don't actually include the information reconciliation and privacy amplification part, so if we wish to use this scheme in real life, we would actually have to perform information reconciliation and privacy amplification, which would further drop the bit rate.

And another fantastic experiment was performed with satellites, where the satellite actually distributed entangled pairs between two ground stations, and the ground stations were one thousand one hundred and twenty kilometers apart. Remember, we said that the light travels in a straight line, whereas using a satellite we can overcome the complication of curved earth's surface to establish a quantum key over much longer distances. So the total distance was over one thousand kilometers, and this measured CHSH violation was 2.56, and the obtained bit rate was 0.12 bits per second. Now, one could ask the question what if we actually use the fiber connecting these two ground stations? Well, the the paper that reported these results estimated that that would have been around eleven orders of magnitude less efficient than using the satellites, which is very, very quite incredible. So this concludes our lessons on quantum key distribution protocols.

\newpage
\begin{exercises}
\exer{Consider the following quantum state:}
\begin{equation*}
\ket{\psi} = \frac{\sqrt{3}}{2}\ket{0} + \frac{1}{2}\ket{1}
\end{equation*}
\subexer{Find the probability of measuring a zero.}
\subexer{Find the probability of measuring a one.}


\end{exercises}

\newpage
\section*{Quiz}
  \addcontentsline{toc}{section}{Quiz}

% \section{Learning more}

\section*{Further reading for Lessons 8-10}
  \addcontentsline{toc}{section}{Further reading for Lessons 8-10}

Lesson 8

This Lesson introduced one the most fundamental protocols of quantum communication. “Mike \& Ike” Chapter 1 goes through the mathematics of the protocol. The original paper is a great read and we highly recommend it:

Charles H. Bennett, Gilles Brassard, Claude Crépeau, Richard Jozsa, Asher Peres, William K. Wootters, Teleporting an unknown quantum state via a dual classical and Einstein-Podolski-Rosen Channels, Physical Review Letters 70, 1895 (1993).

Lesson 9

An enlightening discussion of the BB84 protocol can be found in Chapter 12 of “Mike \& Ike” along with some exercises that will deepen your understanding of the protocol.
The original paper can be found here:

Charles H. Bennett, Gilles Brassard, Quantum cryptography: Public key distribution and coin tossing, Theoretical Computer Science 560, 7 (2014).

Lesson 10

Entanglement-based QKD was first introduced in:

Artur K. Ekert, Quantum cryptography based on Bell’s theorem, Physical Review Letters 67, 661 (1991).

This paper is unfortunately behind a paywall but you should be able to access it through your university’s library system.
Brief discussion of entanglement-based QKD can be also found in Chapter 12 of “Mike \& Ike”.
